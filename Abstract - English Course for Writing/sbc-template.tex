\documentclass[12pt]{article}

\usepackage{sbc-template}

\usepackage{graphicx,url}
\usepackage{pdfpages}
\usepackage{float}
\usepackage[brazil]{babel}   
%\usepackage[latin1]{inputenc}  
\usepackage[utf8]{inputenc}  
% UTF-8 encoding is recommended by ShareLaTex

\usepackage{float}
     
\sloppy

\title{Placidus: A Platform of formal verification for software defined networks}

\author{Levindo Gabriel Taschetto Neto\inst{1}}

\address{IPVS -- University of Stuttgart
  \\
  Stuttgart -- Baden-Württemberg -- Germany
  \email{levindogtn@gmail.com}
}

\begin{document} 

\maketitle

\begin{abstract}
Formal verification is an important step in verifying network operations and ensuring properties, for instance, the accuracy of device configurations. Nevertheless, historically, formal verification techniques are limited by performance, scalability, and expressiveness, mainly due to the complexity of the current networks.
Problems processing time of a task or function in computer networks may be critical for the reliability of a whole system.
In order to solve these possible complications, sophisticated techniques for formal verification can be used, such as (i) updating the information about the context and state of network devices, and (ii) using intelligent strategies for combining representation models for guaranteeing performance and scalability.
Architectures for software defined networks (SDN) offer the essence needed for resolving this kind of issue. It enables combined selection techniques in order to increase the quality of the results obtained by a selection system.
In this paper, we propose the Placidus framework, which is a platform of formal verification for software defined networks. It is based on two techniques for formal verification: header space analysis and exhaustive search to certify the consistency of routing rules in switches, which has a complexity of \textit{O(n)}.
The framework is flexible and capable of verifying a large range of network properties, such as reachability, redundancy in network devices, and inconsistencies.
Preliminary results indicate that the Placidus framework performs well in comparison with other frameworks from its family, is able to verify properties like rule conflict and reachability taking around 5.9 seconds, in a network with more than 6100 traffic flows.
\end{abstract}

\end{document} 